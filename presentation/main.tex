\documentclass[aspectratio=169]{beamer}
%
% Choose how your presentation looks.
%
% For more themes, color themes and font themes, see:
% http://deic.uab.es/~iblanes/beamer_gallery/index_by_theme.html
%
\mode<presentation>
%{https://writelatex.s3.amazonaws.com/vpzdvfpgmtcs/page/20577db5fb5206cebb645db53626924bd7e5686f.jpeg}
  \usetheme{Frankfurt} % or try Darmstadt, Madrid, Warsaw, ...
  \usecolortheme{beaver} % or try albatross, beaver, crane, ...
  \usefonttheme{default} % or try serif, structurebold, ...
  \setbeamertemplate{navigation symbols}{}
  \setbeamertemplate{caption}[numbered]{}
\setbeamertemplate{footline}[frame number]

\usepackage[english]{babel}
\usepackage[utf8x]{inputenc}
\usepackage{hyperref}
\usepackage{graphicx}
\usepackage{moreverb}

\title[GetL_Presentation]{Grammaires et Langages : Présentation}
\author{Equipe Minizza - H4111}
\institute{INSA de Lyon}
\date{Mardi 1 Avril}

\begin{document}

\begin{frame}
\titlepage
\end{frame}

% Plan proposé :
%
% 1.


\section{Structures de données}
\subsection{Diagramme de classes}
\begin{frame}{Diagramme de classes}
\begin{center}
 \includegraphics[scale=0.17]{diagdcla}
\end{center}
\end{frame}

\subsection{Diagramme de classes}
\begin{frame}{Diagramme de classes}
\begin{center}
  \includegraphics[scale=0.3]{ddc_doc}
\end{center}
\end{frame}

\subsection{Diagramme de classes}
\begin{frame}{Diagramme de classes}
\begin{center}
 \includegraphics[scale=0.3]{ddc_ent}
\end{center}
\end{frame}

\subsection{Diagramme de classes}
\begin{frame}{Diagramme de classes}
\begin{center}
  \includegraphics[scale=0.3]{ddc_noeud}
\end{center}
\end{frame}

\subsection{Diagramme de classes}
\begin{frame}{Diagramme de classes}
\begin{center}
  \includegraphics[scale=0.3]{ddc_abs_elmt}
\end{center}
\end{frame}

\subsection{Diagramme de classes}
\begin{frame}{Diagramme de classes}
\begin{center}
  \includegraphics[scale=0.3]{ddc_elmt_burne}
\end{center}
\end{frame}

\subsection{Diagramme de classes}
\begin{frame}{Diagramme de classes}
\begin{center}
  \includegraphics[scale=0.3]{ddc_iaff}
\end{center}
\end{frame}

\subsection{Diagramme de classes}
\begin{frame}{Diagramme de classes}
\begin{center}
 \includegraphics[scale=0.17]{diagdcla}
\end{center}
\end{frame}

\section{Algorithme de validation}
\subsection{Algorithme de validation}
\begin{frame}[fragile]{Algorithme de validation}
\textbf{Objectif :}
Cet algorithme permet de valider un document XML par rapport à un schéma XSD. \\
\textbf{Principe :} algorithme en trois étapes :
\begin{itemize}
\item création d'un dictionnaire noeud/regex
\item passage dans le dictionnaire pour traduire les références
\item comparaison des regex avec la structure du document XML
\end{itemize}
\end{frame}

\section{Algorithme de transformation}

\subsection{Algorithme de transformation}
\begin{frame}[fragile]{Algorithme de transformation}
\textbf{Objectif :}
Cet algorithme permet de transformer un document XML par rapport à une feuille de style XSL. \\
\textbf{Principe :}
\begin{itemize}
\item passage récursif dans chaque noeud du document XSL
\item traitement en fonction du type de noeud
\item sortie du résultat
\end{itemize}
\end{frame}

\begin{frame}[fragile]{Algorithme de transformation}
\textbf{Fonction :}
Selon le type du noeud, l'algorithme sera différent. \\
\textbf{Algorithme pour un élément noeud  :}\\
String Transformation(noeudXMLassocié, sortie)\\
  1) si noeudCourant est value-of\\
  2) si noeudCourant est fore-each\\
  3) si noeudCourant est apply-template avec attributs\\
  4) si noeudCourant est apply-template sans attributs\\
  5) si noeudCourant est template\\
  6) sinon\\
retourner sortie
\end{frame}

\section{Bilan du projet}

\subsection{Bilan quatitatif}
\begin{frame}{Bilan quantitatif}
\begin{center}
 \includegraphics[scale=0.5]{chargeseance}
\end{center}
\end{frame}

\subsection{Bilan quatitatif}
\begin{frame}{Bilan quantitatif}
 \begin{center}
 \includegraphics[scale=0.5]{chargetype}
\end{center}
\end{frame}

\subsection{Bilan qualitatif}
\begin{frame}{Bilan qualitatif}
Montées en compétence :
\begin{itemize}
 \item Modélisation non triviale
 \item Découverte de la syntax \textit{Bison}
 \item (Re)découverte du C++
 \item Implémentation peu aisée
 \item Analyse des langages XML, XSD, XSL
\end{itemize}
\end{frame}

\subsection{Bilan qualitatif}
\begin{frame}{Bilan qualitatif}
Ressenti vis à vis du projet :
 \begin{itemize}
  \item Framework de test génial, très agréable !
  \item Aucun souci avec ce qui nous a été donné
  \item Un projet bien dimensionné
  \item Un projet intéressant
 \end{itemize}
\end{frame}



\end{document}
